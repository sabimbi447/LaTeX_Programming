\pagebreak





\begin{abstract}

\noindent \textbf{Title:} The acceptability of injectable contraception among health care professionals in public health institutions. \\
\noindent \textbf{Author:} OTORO Rejoice Osaretin\\ 
\noindent \textbf{Thesis Supervisor:} Prof. DERRAJI Soufiane  \\
\noindent \textbf{Keywords:} Acceptability - Injectable contraception - Public health institutions - Healthcare professionals \\

\noindent \textbf{Objective:} The main objective of this study is to evaluate the acceptability of injectable contraception among healthcare professionals in public health institutions.\\ 

\noindent \textbf{Methods:} This is a descriptive and retrospective study conducted in four cities of Morocco over a period of 2 months via a questionnaire. The data collected included the most prescribed method of contraception, advantages, disadvantages, acceptability, adherence, side effects, failure and potential causes of failure of injectable contraceptives. We also asked whether this method was proposed to women, if they were followed up and their satisfaction level with the use of this method. \\

\noindent \textbf{Results:} A total of 40 questionnaires were studied. The results revealed that, although contraceptive pills were the most prescribed, the majority of health professionals suggested and followed injectable contraception. Although the majority recognized the effectiveness and advantages of this method, concerns were expressed about its disadvantages and side effects, resulting in its low acceptability, despite most users of this method being satisfied with it. \\
 
\noindent \textbf{Conclusion:} This study sheds light on the acceptability of injectable contraception in public health institutions and proposes avenues for improvement. Collaborating among all health care stakeholders, is necessary to improve its acceptability, as it not only serves as an effective form of contraception but also as a long-term contraceptive and most especially a solution, addressing the challenge of forgetting daily contraceptive pills. 

\end{abstract}



%%%%%%%%%%%%%%%%%%%%%%%%%%%%
%%%% 

\pagebreak

\includepdf[pages=-,pagecommand={\thispagestyle{fancy}}]{C:/Users/SABIMBI/Desktop/Rejoice Final Year Project/arabic_page.pdf}


\pagebreak

\begin{itemize}
  \item[\textcolor{white}{$\Box$}] 
\end{itemize}


\vspace{7cm}

\begin{figure}[H]
  \includegraphics{Images/annexes.png}
\end{figure}

\section*{}
\addcontentsline{toc}{section}{ANNEXES}


\pagebreak

\fontsize{18}{15}\selectfont % Change the numbers as needed (12pt font with 15pt baseline skip)
\noindent \textbf{Questionnaire sur l’acceptabilité de la contraception \newline injectable dans les établissements de santé publique auprès des professionnels de santé}

\fontsize{16}{20}\selectfont % Change the numbers as needed (12pt font with 15pt baseline skip)


\begin{enumerate}[label=\arabic*.]
  \item	Quel est votre statut professionnel actuel ? 
  \begin{itemize}[label=$\square$]
      \item Médecin
      \item Infirmier 
      \item Sage–femme  
      \item Autres : 
  \end{itemize}
  \vspace{2em}

  \item Où exercez-vous actuellement votre profession ? 
    \begin{itemize}[label=$\square$]
      \item Dispensaire 
      \item Centre de santé 
      \item Centre mère-enfant   
      \item Hôpital régional 
      \item CHU
      \item Autres : 
    \end{itemize}
    \vspace{2em}
  
  \item Zone d’exercice : 
    \begin{itemize}[label=$\square$]
      \item Zone urbaine 
      \item Zone rurale     
    \end{itemize}
    \vspace{2em}
  
  \item	Parmi les méthodes contraceptives existantes, laquelle est plus prescrite ? 
    \begin{itemize}[label=$\square$]
      \item La contraception orale (pilule)
      \item La contraception injectable 
      \item Autres : 
    \end{itemize}
    \vspace{2em}

  \item	Proposez-vous la contraception injectable aux patientes ? 
    \begin{itemize}[label=$\square$]
      \item Oui 
      \item Non 
    \end{itemize}
    \vspace{2em}

  \item	À votre avis, quels sont les avantages de la contraception \newline injectable 
    \begin{itemize}[label=$\square$]
      \item Surmonte l’oubli 
      \item Longue durée de contraception 
      \item Moins d’effets indésirables 
      \item Autres : 
    \end{itemize} 
    \vspace{2em}

  \item	À votre avis, quels sont les inconvénients de la contraception injectable ? 
    \begin{itemize}[label=$\square$]
      \item Risque d’infection du site d’administration
      \item Irréversibilité dans la durée prescrite
      \item Plus d’effets indésirables
      \item Autres : 
    \end{itemize} 
    \vspace{2em}

  \item Acceptabilité de la contraception injectable:
    \begin{itemize}[label=$\square$]
      \item Très acceptable 
      \item Peu acceptable 
      \item Non acceptable 
    \end{itemize}
    \vspace{2em}

  \item Adhérence à la contraception injectable ? 
    \begin{itemize}[label=$\square$]
      \item Oui 
      \item Non 
    \end{itemize}
    \vspace{2em}

  \item Quels sont les effets indésirables signalés par les patientes sous contraception injectable : 
    \begin{itemize}[label=$\square$]
      \item Troubles menstruels 
      \item Prise de poids 
      \item Baisse de la libido 
      \item Réaction au site d’injection 
      \item Hypersensibilité mammaire 
      \item Hirsutisme (la pilosité) 
      \item Autres : 
    \end{itemize} 
    \vspace{2em}

  \item Avez-vous reçu des cas d’échec de la contraception injectable ? 
    \begin{itemize}[label=$\square$]
      \item Oui 
      \item Non 
    \end{itemize}
    \vspace{2em}

  \item Si oui, que pensez-vous être la cause de l’échec ?
    \begin{itemize}[label=$\square$]
      \item La tolérance 
      \item Non-respect du délai d’abstinence 
      \item Enceinte avant de prendre l’injection 
      \item Non-respect de la procédure clinique 
      \item Autres :
    \end{itemize} 
    \vspace{2em}

  \item Y a-t-il un suivi pour les patientes sous contraception injectable ?  
    \begin{itemize}[label=$\square$]
      \item Oui 
      \item Non 
    \end{itemize} 
    \vspace{2em}

  \item	Vos patientes sous contraception \newline injectable sont-elles : 
    \begin{itemize}[label=$\square$]
      \item Satisfaites 
      \item Peu satisfaites 
      \item Non satisfaites
    \end{itemize} 
\end{enumerate}