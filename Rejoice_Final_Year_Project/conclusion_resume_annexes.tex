\pagebreak

\begin{itemize}
  \item[\textcolor{white}{$\Box$}] 
\end{itemize}


\vspace{7cm}

\begin{figure}[H]
  \includegraphics{Images/conclusion.png}
\end{figure}

\section*{}
\addcontentsline{toc}{section}{CONCLUSION}



\pagebreak


\noindent Notre étude nous a permis d’explorer l’acceptabilité de la contraception injectable à travers le regard des professionnels de santé dans les centres de santé de quatre villes du Maroc. Les résultats ont mis en évidence la nécessité d'une action accrue de la part des acteurs de la santé pour améliorer l'acceptabilité de cette méthode, qui, malgré son ancienneté, demeure encore peu acceptée par la majorité de la population.\\

\noindent Bien que les résultats indiquent une faible acceptabilité, il est notable que la majorité des utilisatrices expriment leur satisfaction à l'égard de cette méthode contraceptive. Cependant, il est impératif que les professionnels de la santé prennent l’initiative de sensibiliser les femmes sur les avantages de cette approche tout en mettant en évidence les éventuels effets secondaires en particulier l’aménorrhée qui est la principale raison de la discontinuité de cette méthode. \\

\noindent L'éducation et la sensibilisation jouent un rôle crucial dans la promotion de l'acceptabilité des méthodes contraceptives, en particulier en ce qui concerne la contraception injectable, l'intégration de la contraception injectable dans les programmes des formation médicale est essentielle car, tous les professionnels de santé inexpérimentés ont été incapables de répondre aux questions sur les contraceptifs injectables.\\ 

\noindent Notre recherche a mis en lumière un obstacle majeur à la prescription et à l'utilisation généralisée des contraceptifs injectables, à savoir la non-disponibilité de ces produits. Il est impératif de résoudre le problème des ruptures de stock, en veillant à ce que ces contraceptifs soient accessibles et disponibles dans toutes les structures de santé. Une autre constatation concerne le suivi des utilisatrices de cette méthode contraceptive. Il est impératif de souligner l'importance du suivi régulier, car cela renforce l'adhérence.\\

\noindent Les recherches futures devraient élargir leur échantillon en visant une population plus diversifiée, notamment dans les zones rurales, et explorer les disparités d'acceptabilité entre les zones urbaines et rurales.\\

\noindent Enfin, une compréhension approfondie de l'acceptabilité de la contraception injectable du point de vue des professionnels de la santé ainsi que la prise en compte des lacunes identifiées sont cruciales pour optimiser son acceptabilité et son utilisation dans les établissements de santé publique. Cela contribuera à améliorer les services de planification familiale et de santé reproductive, renforçant ainsi l'ensemble du système de soins de santé.





%%%%%%%%%%%%%%%%%%%%%%%%%%%%%%%%%%%%%%%%%%%%%%
%%%%%%%%%%   Resume   %%%%%%%%%%%%%%%%%%%%%%%%%


\pagebreak


\begin{itemize}
  \item[\textcolor{white}{$\Box$}] 
\end{itemize}


\vspace{7cm}

\begin{figure}[H]
  \includegraphics{Images/resumes.png}
\end{figure}

\section*{}
\addcontentsline{toc}{section}{RESUME}

\pagebreak


\begin{center}
  \textbf{RESUME}
  
\end{center}

\noindent \textbf{Titre :} L’acceptabilité de la contraception injectable au niveau des établissements de santé publique auprès des professionnels de santé.  

\noindent \textbf{Auteur :} OTORO Rejoice Osaretin \\
\textbf{Directeur de la thèse :} Pr. DERRAJI Soufiane.  \\
\textbf{Mots clés :} Acceptabilité - Contraception injectable - Etablissements de santé publique - Professionnels de santé. \\

\noindent \textbf{Objectif :} L’objectif principal de cette étude est l’évaluation de l’acceptabilité de la contraception injectable dans les établissements de santé publique auprès des professionnels de santé. \\

\noindent \textbf{Méthodes :} Il s’agit d’une étude descriptive et rétrospective réalisée dans les centres de santé de quatre villes au Maroc pendant une période de 2 mois via un questionnaire. Les données recueilles incluent la méthode de contraception la plus prescrite, les avantages, les inconvénients, l’acceptabilité, l’adhérence, les effets indésirables, l’échec, les causes potentielles d’échec des contraceptifs injectables. Nous avons également demandé si cette méthode était proposée aux femmes, si elles étaient suivies et leur degré de satisfaction quant à l’utilisation de cette méthode. \\
   
\noindent \textbf{Résultats :} Au total, 40 questionnaires ont été étudiés. Les résultats ont révélé que même si les pilules contraceptives sont les plus prescrites (92,5\%), la majorité des professionnels de santé (82,5\%) ont suggéré la contraception injectables aux femmes et les ont suivies. Bien que la majorité reconnaisse l’efficacité et les avantages de cette méthode, des inquiétudes ont été exprimées concernant ses inconvénients et ses effets indésirables et, par conséquent, elle reste peu acceptable (77,5\%), même si la plupart des utilisatrices de cette méthode en sont satisfaites. \\
   
\noindent \textbf{Conclusion :} Cette étude met en lumière l’acceptabilité de la contraception injectable dans les établissements de santé publique et propose des pistes d’amélioration. Une collaboration de tous les acteurs de santé est nécessaire pour améliorer son acceptabilité car il s’agit non seulement d’une contraception efficace, mais aussi d’une contraception à long terme et surtout une solution pour surmonter le défi de l’oubli des pilules contraceptives journalières.  


